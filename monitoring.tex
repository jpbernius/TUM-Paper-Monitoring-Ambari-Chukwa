\documentclass{acm_proc_article-sp}
\usepackage[english,ngerman]{babel}
\usepackage[utf8]{inputenc}
\usepackage{abbrevs}
\usepackage{url}

\usepackage[
pdftex,colorlinks=false,
bookmarks=true,b
ookmarksopen=true,
bookmarksopenlevel=0,
plainpages=false,
bookmarksnumbered=true,
hyperindex=false,
pdfstartview=,
pdfauthor={Jan Philip Bernius, Ekaterina Sebina},
pdftitle={Monitoring: Ambari and Chukwa}
]{hyperref}


%% ABBREVIATIONS
\newabbrev\isds{Internet-scale Distributed Systems (ISDS)}[ISDS]
\newabbrev\amb{Apache \textit{Ambari}}[\textit{Ambari}]
\newabbrev\chuk{Apache \textit{Chukwa}}[\textit{Chukwa}]
\newabbrev\hadoop{Apache Hadoop}[Hadoop]
\newabbrev\hdfs{Hadoop Distributed File Systems (HDFS)}[HDFS]
\newabbrev\mr{MapReduce (MR)}[MR]
\newabbrev\demux{de-multiplex (demux)}[demux]


\begin{document}
\title{Monitoring: Ambari and Chukwa}
\subtitle{Seminar: Internet-scale Distributed Systems}

\numberofauthors{2}

\author{
% 1st. author
\alignauthor
Jan Philip Bernius\\
       \affaddr{Technische Universität München}\\
       \affaddr{Department of Computer Science}\\
       \email{\href{mailto:janphilip.bernius@tum.de}{janphilip.bernius@tum.de}}
% 2nd. author
\alignauthor
 Ekaterina Sebina\\
       \affaddr{Technische Universität München}\\
       \affaddr{Department of Computer Science}\\
       \email{\href{mailto:katie.sebina@tum.de}{katie.sebina@tum.de}}
}

\date{15 July 2015}

\maketitle

%TODO REMOVE!
\nocite{*}

\begin{abstract}
%TODO: Abstract

The main topic of this paper is the concept of monitoring in distributes systems presented in \ambshort and \chukshort. 
Our thesis is to explain the core difference between \ambshort and \chukshort, based on the monitoring process and find the most suited environment for each application. 
The emphasis is on the benefits that each application is offering to the current technology processes and systems. 
With an short introduction on the architectures of both application, the core field of use can be explained. 
\ambshort's main function is to simplify the use of Hadoop, while \chukshort's is to optimise the analyse and storage on Hadoop clusters. 	
\end{abstract}

\category{H.3.4}{Systems and Software}{Distributed systems}
\category{C.2.3}{Network Operations}{Network monitoring}


\terms{Documentation, Experimentation, Measurement, Performance, Reliability, Security, Standardization, Monitoring}

\keywords{Hadoop, Monitoring, Map and Reduce, Ambari, Chukwa, Apache, OpenSource, Java, Internet-Scale, Distributed Systems, Cluster, Cloud}

\noindent\rule{84mm}{0.4pt}

\section{Introduction}
%TODO: Introduction
As Internet Applications get bigger and bigger, more and more servers get involved in the system and Applications will run on multiple Servers on parallel. Further the number of running nodes will vary based on user number and activity. As these systems get impossible to administer by hand, automating the process is very important. Systems tend to fail from time to time, therefore it is very important to notice failure and fix occurring problems.


\subsection{Definitions}

\textit{Monitoring} is a process, which controls, interacts and manages other processes. It has the ability to locate, prevent and repair failures. It is an important strategy for availability and security.

\amb is an open source framework that allows to control, manage, provide and install Hadoop Cluster easily from the included Web-Interface.

\chuk is a large-scale Log Collection and Management System. It is specialised to collect Log files as well as Application Metrics from distributed Clusters.

\subsection{Structure of this paper}
First, the concepts of Hadoop and Map Reduce~(\ref{subsec:Hadoop}) are explained, these concepts are the basis of the techniques introduced in this paper. Second, the concept of monitoring~(\ref{subsec:ConceptMonitoring}) is introduced and two interpretations and implementations, Apache Systems \amb and \chuk are explained~(\ref{subsec:Implementation}) in more detail. Thirdly, a comparison~(\ref{subsec:Differentiation}) is made and, lastly, a conclusion~(\ref{sec:Conclusion}) is drawn.

\subsection{Literature Review}
	As it is highly recommended to document the research and review process,~\cite{brocke09} this paragraph will summarise the research and review process.
	
	The research started on the self description on the project websites and continued using technical documentation, describing the underlying architecture of both systems.
	Further, Library Databases, especially \emph{IEEE Xplore} and \emph{EBSCOhost} were utilised to find more in-depth material, including the research these projects are based on.
	
\subsection{Hadoop}
\label{subsec:Hadoop}
Hadoop is an open source software by Apache for storing and analysing data. It divides and storages Big Data across multipel smaller clusters, by copying them three times. This concept implies a high fault tolerance, if one of the clusters fail.
 \\
 Hadoops Architecture is divided in many parts. To understand \amb and \chuk, the three of the following parts will be enough.
  \\
  1.Hadoop Clusters are multipel new storage places for the data. They are divided into one master node and several server nodes.
  \\
  2.Hadoop Distributed File Systems (HDFS), are file systems that are responsible for storing data across multipel machines.
  \\
  3.Map and Reduce is the main function which analyses and manages data on Hadoop Clusters.
\subsubsection*{Map Reduce}
The Map and Reduce process is the main function of Hadoop. After the division of data across multipel clusters, it maps similar datasets together into a new cluster. Later on it analyses and reduces the data till the main sense. This data is getting combined together into the output. 
\\
Hadoop Users need to implement the rules for the map and reduce process. All the other parts of the function such as sort and store are already available, when Hadoop is getting installed.


\amb and \chuk both presuppose a basic understanding of Hadoop as well as the map and reduce function. Both application are powered by Apache and are monitoring processes for Hadoop.They differ in their area of operation of monitoring. 

\section{Ambari and Chukwa}
The intension is to show the advantages of each application with in a use case to answer the question for with monitoring operation \amb or \chuk should be preferred.
\subsection{What is the concept of monitoring and how is it used in general?}
\subsection{How is the concept of monitoring implemented in Ambari and Chukwa?}
Ambari is an open spurce framework that allows the User to monitore, manage and install Hadoop. 
\\
Ambari can be divided in two parts. First the Amabri Web which is the main plattform for users to log in and give up request that should be run throug Ambari. Secondly the Amabri Server, which is divided into smaller parts itself. The API or REST API is connected to different Web applications, the most important one the Ambari Web. Other interfaces like Microsoft System Center, are applications that allos the user to analyse the data forthermore the capabilietie of Hadoop. Amabri Server installs on each cluster an Ambari Agent, that gives every few seconds a heartbeat to the server; the server answers with an instruction for the Agengt or it sends a confermation about his current lifestatus. 
\\
Garaphic 
\\
To explain the process of monitoring by Ambari we wrote our own scenario. Steps Amabri will fullfile if one the Hadoop Clusters fails.
\\
Essencially, if one of the Clusters fail, the Agent on it would be movend into the inactiv state. This means it would not be able to send an heartbeat to the Ambari server. Amabri server would noticed this, in the worst case scenario, in two secondes, and start an repairing process. The User does not have to worry about the data that was on this broken cluster, because of the map and reduce prinzip there are two more copies of this data in our distributed system. NACHLESEN NOCHMAL 
\subsection{How does Ambari differentiate from Chukwa?}
\label{subsec:Differentiation}

\amb and \chuk are, while both being Monitoring Systems, engineered for different use-cases. 
As stated in \ref{subsec:ConceptMonitoring}, the concept of monitoring allows different applications in the domain of \isds. 

\subsubsection{Ambari}
\amb was founded to simplify the use of Hadoop Systems. It offers management, monitoring and intervening process on Hadoop Clusters.\cite{Hortonworks2013} 

After-all, \amb fulfils two tasks on a \hadoop cluster. 
First, \amb provides metrics about a cluster in general, like number of running hosts, cluster health, overal RAM/CPU Usage, Network Traffic etc. \amb also provides Host specific Metrics.
\\
Second, \amb can perform Actions on the entire Cluster, a group of Hosts or individual hosts. 
%TODO extend on actions

\subsubsection{Chukwa}
\chuk offers extensive data collection capabilities, especially optimised for storing, archiving and analysing ``a large and open-ended set of time series metrics and logs.''~\cite{Boulona}

Different to \amb, \chuk does only perform read operations on the hosts. Its highly configurable Agent will upload all data to the Collector, where processing takes place.


Even though some Use Cases~(\ref{netflix}) show that it is possible, \chuk does not aim to provide Real-Time information. However, after processing is done, the data is available in a structured form and can be stored forever. \mr and \demux also allows automated analysis and error detection as shown in the Use Case~(\ref{netflix}). This allows reporting to responsible System Administrators or Software- Engineers in a efficient manner. It also provides detailed error information, as the related log information is available.

\subsubsection{Comparison}
%TODO rethink if it fits this way
In comparison of both applications, \amb approaches a wilder spectrum trough out of all Hadoop tasks, while \chuk is specialised on storing and analysing \hadoop Data. Both differ in their core range of use for \hadoop, in connection of monitoring. \amb is not capable of monitoring the actual data value that is stored on the clusters that are getting observed, but it can monitor the health of those.\cite{ApacheSoftwareFoundation2015} While \chuk has the ability to interpreter and analyse data on its monitored clusters.

Both Systems offer \emph{complementary} services. \amb can do real-time management of the cluster and provide real-time system status. \chuk will archive system- and Application-Log-Information and put it in structured \hdfs storage.

\section{Conclusion}
\label{sec:Conclusion}

In dependency of the Use Case on monitoring it is recommended to use Ambari or Chukwa. Both application were founded by Apache, which is also the publisher of Hadoop. In eather way each of them fullfiels there range of use for monitoring processes properly. 
\subsection{Summary}

\subsection{Recommended Use Cases}
One of the most known users of a \chuk derivate is \textbf{Netflix}. 

\label{netflix}
\begin{figure}[hbt]
  \centering
  % https://drive.google.com/open?id=1Nmn-mWvxAwhcB9Vo4nbq0gutnWp0bPcCfBU14KMDVQ8
  \includegraphics[width=\linewidth,clip=true,trim=5mm 2cm 0 5mm]{images/NetflixSuro}
  \caption{Netflix Suro Architecture~\cite{Bae2013, Harris2013}}
  \label{fig:SuroArchitecture}
\end{figure}


\subsection{Feature Research Questions}
Current research has not shown, in which extend the two systems \amblong and \chuk can be used in an inter-connected manner, complementing each others services. Managing large, internet-scale clusters, consisting of multiple thousand hosts running \hadooplong can be more automated and controlled with Systems like \amb and \chuk in place, not only controlling the current state, but also managing installed components and archiving system-state information.

Further, an improved way of storing Chukwas log data in a more structured way, allowing live querying, is still be to found. Boulon et al. did look into several options so far~\cite{Boulonb}, but this work has to be continued.

Last, but not least, the real-time approach shown in \ref{netflix} may be extended, as the ``recent trend has been in the area of real-time stream processing''~\cite{Bae2013} and will most likely continue in the future.

\bibliographystyle{abbrv}
\bibliography{bibliography}

\end{document}
